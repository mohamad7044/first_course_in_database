\documentclass{article}

\title{Relational Aljebra}
\author{ts2964}

\usepackage{fontspec}
\defaultfontfeatures{Scale=MatchLowercase}
\setmainfont[Mapping=tex-text]{Cambria}
\setsansfont[Mapping=tex-text]{Cambria}
\setmonofont{Courier New}

\usepackage{xltxtra}
\usepackage{unicode-math}
\usepackage{latexsym}
\usepackage{array}

\begin{document}

\maketitle

\section{Introduction}
Relational aljebra are a set of operation on relations/tables. Recalled that in integer like 1,2,3,4,5, etc., we have operation like $+,-,\times, \div $.
Similarly, in relations/tables, we have operations; union $\cup$, intersection $\cap$, differences, cartesian products $\times$, join $\Join$. There need to use symbols make it difficult to use ascii-based text editor to write notes.

Given a relation $R$ ( a table with data called $R$), we could use relational aljebra operation on the relation $R$. These operations are union, intersection, differences, cartesian and join.

In  the text, it uses these sysmbols;
\begin{enumerate}
\item $\pi$ for rows operation, select certain rows
\item $\sigma$ for column operation, choose certain attributes
\end{enumerate}

For instance, see example on page 43-47.
The relational aljebra can be combined to for arbitrary complex expresson.

\section{Question 2.4.1}

Recalled that, there are 4 relations. The common attributes in all relations/tables are model. We would use ra, mysql/sqlite3 for answers.

(a) Choose \textit{maker} that sell \textit{printer}, but not {pc}. We need \textit{maker}, thus it is a projection. We only need \textit{printer} but not \textit{pc}, thus a selection. The relation we would use is \textit{Product}.

Let $Ri$ where $i=1,2,3,..$ be a temporary relation/table in our computation.
\begin{eqnarray}
R1 &:=&  (\pi_{maker})(\sigma_{type=printer})(Product) \\
R2 &:=&  (\pi_{maker})(\sigma_{type=pc})(Product) \\
R3 &:=& (\pi_{maker})(R1 - R2) 
\end{eqnarray}


R1 find all maker that sell printer (D,E, H), R2 find all maker that sell pc (A, B, C, D, E). Therefore, those who sell printer, but not pc is (R1-R2) which is (H).
 
SQL: 
\begin{verbatim}
CREATE VIEW [U] AS 
select distinct maker from Product where type="pc";

CREATE VIEW [V] AS 
select distinct maker from Product where type="printer";

select maker from V where maker not in ( select maker from U);
\end{verbatim}

(b) what pc model have speed at least 2.5? (1001,1004,1005,1006,1010,1012,1013)
\begin{eqnarray}
R1 &:=& (\sigma_{speed} \geq 2.5 )(PC) \\
R2 &:=& (\pi_{maker})(R1)
\end{eqnarray}

SQL: SELECT maker from PC where speed >= 2.5

(c) What maker produced laptop with hd atleast 120GB? (A,B,E,G)
\begin{eqnarray}
R1 &:=& (\sigma_{hd>120})(LapTop) \\
R2 &:=& (\pi_{maker})(R1) \\
R3 &:=& (Product)\Join(R2)
\end{eqnarray}

SQL;
\begin{verbatim}
select maker from Product where model in 
(select model  from LapTop where hd >= 120)

select Product.maker from Product, LapTop where 
Product.model=LapTop.model AND LapTop.hd >= 120;
\end{verbatim}

(d)Find the model and price for all product by maker C. (
1007, 510)
\begin{eqnarray}
R1 &:=&  (\sigma_{maker=C})(Product \Join PC)\\
R2 &:=&  (\sigma_{maker=C})(Product \Join LapTop)\\
R3 &:=&  (\sigma_{maker=C})(Product \Join Printer)\\
R4 &:=&  (\pi_{model,price})(R1) \\
R5 &:=&  (\pi_{model,price})(R2) \\
R6 &:=&  (\pi_{model,price})(R3) \\
R7 &:=& R4 \cup R5 \cup R6 
\end{eqnarray}

SQL:
\begin{verbatim}
select maker,Product.model from Product JOIN 
   LapTop on Product.model = LapTop.model 
   where Product.maker="C";
select maker,Product.model from Product JOIN 
   PC on Product.model = LapTop.model 
   where Product.maker="C";
select maker,Product.model from Product JOIN 
  Printer on Product.model = LapTop.model 
  where Product.maker="C";

select model, price from LapTop where model in 
(select model from Product where maker="C") UNION 
select model,price from PC where model in 
(select model from Product where maker = "C" )
\end{verbatim}

(e) find model of black-and-white printer. (3002, 3005 )
\begin{eqnarray}
R1 &:=& (\sigma_{color=0})(Printer) \\
R2 &:=& (\pi_{model})(R1)
\end{eqnarray}

SQL:
\begin{verbatim}
select model from Printer where color=0;
\end{verbatim}

(f) find hd sizes that occur in 2 or more PC.(80(2),160(2),250(6))



SQL:
\begin{verbatim}
CREATE VIEW W SELECT hd,count(hd) as hdc from PC group by hd
select * from W where hdc >= 2

select * from (select hd,count(*) as hdc from PC group by hd ) 
where hdc >=2
\end{verbatim}

(g)find pairs of PC that have same speed and RAM.
(1004,1012)
\begin{eqnarray}
R1 &:=& (\sigma_{speed_{a}=speed_{b},ram_{a}=ram_{b},model_{a} < model_{b}})(PC_a \Join PC_b )
\end{eqnarray}

\begin{verbatim}
select A.model, B.model,A.speed, A.ram, B.speed,B.ram from 
PC as A join PC as B on A.speed = B.speed AND A.ram = B.ram 
AND A.model < B.model
\end{verbatim}

(h) find makers of atleast 2 different pc or laptop with atleast speed 2.2.
(B, E )

SQL: 
\begin{verbatim}
create view  X as select model,speed from PC where 
    speed >=2.2 UNION select model,speed 
    from LapTop where speed >= 2.2

select maker from ( select maker,count(*) as mkc 
   from X group by maker) 
   where mkc >=2 
\end{verbatim}

(k) Find makers with exectly 3 different models PC. ( A,B,D,E)

\begin{verbatim}
select maker from  ( select maker,count(*) as mkc  
    from ( Product JOIN  PC on Product.model = PC.model )  
    group by maker  ) 
    where mkc =3
\end{verbatim}

(j) find makers with the highest speed for PC and LapTop. ( maker=B, model=1005,1006).

\begin{verbatim}
select maker,model from Product where model in  
     (select model from PC where speed in 
     (select max(speed) from 
     ( select model,speed from PC UNION 
     select model,speed from LapTop )))
\end{verbatim}

(l) makers with atleast 3 different speeds. ( A (1.42,2.1,2.66), D(2,2.2,2.8), E(1.86,2.8,3.06)) 

\begin{verbatim}
create view Ri as select Product.maker, PC.model, PC.speed from
      PC JOIN Product on Product.model = PC.model
      # note i = 1,2,3
select R1.maker, R1.speed, R2.speed,R3.speed from R1,R2,R3 
      where R1.maker=R2.maker and R1.maker = R3.maker and 
      R1.speed < R2.speed and R1.speed < R3.speed and 
      R2.speed < R3.speed
\end{verbatim}

\end{document}


% snippet
\begin{eqnarrray}
R1 &:=&  \\
R2 &:=&  \\
\end{eqnarray}